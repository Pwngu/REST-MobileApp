%!TEX root=../document.tex

\section{Ergebnisse}
\label{sec:Ergebnisse}

Die App wurde in iOS mit der Programmiersprache Swift entwickelt.

\subsection{Swift}
\label{sec:Swift}

Um Nochmal Swift zu wiederholen hab ich mir das offizielle Tutorial von Apple angeschaut \cite{swift-tutorial}.


\textbf{Basics} \\
In Swift können im allgemeinen redundante Symbole weggelassen werden; also Strichpunkte und runde Klammern bei \texttt{if}s, sowie \texttt{break}s in \texttt{switch} statements können weggelassen werden (funktioniert dabei ähnlich wie Python).

Swift ist stark typisiert, der Dateityp wird aber vom Compiler inferred; das heißt man kann, aber muss ihn nicht explizit dazu schreiben:

\begin{lstlisting}[caption=Normale Variablendefinition]
var x = 0
\end{lstlisting}

\begin{lstlisting}[caption=Variablendefinition mit definiertem Datentyp]
var str: String = "asdf"
\end{lstlisting}

Konstanten werden mit \texttt{let} definiert. Laut Konvention sollten alle Variablen, welche während des Programmablaufs nicht geändert werden sollten mit \texttt{let} deklariert werden.
Der Compiler warnt auch, wenn er eine \texttt{var} Deklaration findet, welche nicht geändert wird.

\begin{lstlisting}[caption=Definition einer Konstanten]
let const = 123
\end{lstlisting}


\textbf{Funktionen} \\
Funktionen sehen in Swift wie folgt aus:

\begin{lstlisting}[caption=Allgemeine Funktionsdefinition]
func <name>(<parameters>) -> <return type> {
    <function body>
}
\end{lstlisting}

Die Rückgabetypdefinition kann, wenn es keinen Rückgabetyp gibt auch weggelassen werden; \texttt{-> Void} ist also nicht notwendig.


\textbf{Class und Struct} \\
In Swift gibt es einen Unterschied zwischen Klassen und Structs. Structs können prinzipiell mit ihrem Gegenstück in C verglichen werden, sie sind quasi eine Ansammlung an Variablen/Werten und haben keine Funktionen (abgesehen von einer init Funktion).

Klassen hingegen funktionieren wie in anderen Objektorientierten Programmiersprachen auch; Sie können Objekt-Funktionen haben und voneinander erben.
Ein anderer wichtiger Unterschied ist, dass Structs \emph{pass by value} und Klassen \emph{pass by reference} sind.

\begin{lstlisting}[caption=Beispiel Structdefinition]
struct SomeStruct {
    var name: String
    init(name: String) {
        self.name = name
    }
}
\end{lstlisting}

\begin{lstlisting}[caption=Beispiel Klassendefinition]
class SomeClass {
    var name: String
    init(name: String) {
        self.name = name
    }
    
    func someFunc() -> String {
        return self.name
    }
}
\end{lstlisting}

\begin{lstlisting}[caption=Beispiel Subklassendefinition von SomeClass]
class SomeSubClass: SomeClass {
    var subName: String
}
\end{lstlisting}


\subsection{Implementierung}
\label{sec:Implementierung}


